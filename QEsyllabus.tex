\documentclass[11pt]{article}
\usepackage{amsfonts,latexsym,amsthm,amssymb,amsmath,amscd,euscript}
\usepackage{mathrsfs}
\usepackage[top=1in, bottom=0.8in, left=0.8in, right=1in]{geometry}

\setlength{\columnsep}{0.1pc}



\title{Qualifying Exam Syllabus}
\author{Jason Zhao}
\date{13 October 2022, 10 AM -- 1 PM, Evans 961}
\begin{document}

\maketitle

Committee:  Richard Bamler (Exam Chair), Sung-Jin Oh (Advisor), Ruixiang Zhang, Sunčica Čanić
 (Department of Academic Senate Representative).

\section{Major topic:  Partial differential equations (Analysis)}

References: Hormander, \textit{The analysis of linear partial differential operators I}; Evans, \textit{Partial differential equations}; Tao, \textit{Nonlinear dispersive equations: local and global analysis}. 
\begin{itemize}
\item \textbf{Distributions} (Hormander I--IV)\\
Test functions, distributions, tempered distributions, differentiation, multiplication by smooth function, convolution, approximations, fundamental solutions. 

\item \textbf{Four important PDE} (Evans 2) \\
Energy methods, fundamental solutions, Laplace's equation (mean value property, maximum principle, Harnack's inequality), heat equation (maximum principle, smoothing), wave equation (finite speed of propagation, Huygen's principle).

\item \textbf{Sobolev spaces} (Evans 5 and Tao Appendix A) \\
Weak derivatives, Holder spaces, Sobolev spaces, approximations, extensions, traces, Sobolev embedding, compact embedding, Poincare inequality.

\item \textbf{Second-order elliptic equations} (Evans 6) \\
Energy method, existence and uniqueness, Fredholm alternative, local and global $H^k$-elliptic regularity, maximum principles. 

\item \textbf{Calculus of variations} (Evans 8.1, 8.2, 8.4, 8.6) \\
First variation, Euler-Lagrange equations, existence of minimisers, constrained minimisers, Lagrange multipliers, conservation laws. 


\item \textbf{Dispersive equations} (Tao 2.1--2.6, 3.1--3.6) \\
Linear and non-linear Schrodinger and wave equations, notions of solutions, dispersive estimates, Strichartz estimates, monotonicity formulae, $X^{s, b}$ spaces, local existence theory, conservation laws, global existence theory, scattering theory. 
\end{itemize}

\section{Major topic:  Harmonic analysis (Analysis)}

References: Stein, \textit{Harmonic Analysis}; Duoandikoetxea, \textit{Fourier Analysis}; Grafakos, \textit{Classical Harmonic Analysis}; Tao, \textit{Nonlinear dispersive equations: local and global analysis}.
\begin{itemize}
\item \textbf{Function spaces and interpolation} (Grafakos 1.1--1.4)\\
$L^p$-spaces, convolution, Young's inequality, approximations to the identity, Riesz-Thorin interpolation, Lorentz spaces, duality, Marcinkiewicz interpolation, Schur's test.

\item \textbf{Fourier transform} (Duoandikoetxea 1.1--1.9)\\
Fourier coefficients and series,
criteria for pointwise convergence, summability methods and $L^p$-convergence of Fourier series, Fourier transform, Schwartz functions, Fourier inversion, Plancharel theorem, Riemann-Lebesgue lemma, Hausdorff-Young inequality. 

\item \textbf{Maximal functions} (Duoandikoetxea 2.1--2.7, Stein II.1, V.1--V.3) \\
Approximation to the identity, Wiener-Vitali covering lemma, rising sun lemma, Hardy-Littlewood maximal function, Lebesgue differentiation theorem, doubling measures, $A_p$ weights, weighted maximal inequality, vector-valued maximal inequality. 

\item \textbf{Singular integrals} (Duoandikoetxea 5.1--5.4) \\
Calderon-Zygmund decomposition, convolution kernels, general Calderon-Zygmund kernels, truncated integrals and principal values, singular integrals, Fourier multipliers, Mikhlin multipliers.

\item \textbf{Littlewood-Paley theory} (Tao Appendix A) \\
Littlewood-Paley projections, Bernstein and Sobolev-Bernstein inequalities, Hardy-Littlewood-Sobolev inequality, Gagliardo-Nirenberg inequality, Sobolev embedding, square function, characterisations of Holder and Sobolev spaces, fractional product rule, fractional chain rule. 

\item \textbf{Oscillatory integrals} (Stein VIII.1--VIII.2) \\
Oscillatory integrals of the first kind, non-stationary phase, van der Corput lemma, stationary phase, dispersive estimates.

\end{itemize}

\section{Minor topic: Smooth manifolds (Geometry)}

References: Lee, \textit{Introduction to Smooth Manifolds}; Warner, \textit{Foundations of Differentiable Manifolds and Lie Groups}
\begin{itemize}
\item \textbf{Manifolds} (Lee 1--6) \\
Smooth structures, smooth functions, partitions of unity, sub-manifolds, immersions, submersions, embeddings, Sard's theorem, Whitney's embedding theorem. 

\item \textbf{Vector bundles} (Lee 8--11) \\
Vector fields, Lie brackets, flows, Lie derivatives, time-dependent vector fields, tangent and cotangent bundles, differentials. 

\item \textbf{Differential forms and integration} (Lee 12, 14--16)\\
Tensors, wedge products, $k$-forms, pullbacks, exterior derivative, closed and exact forms, Poincare's lemma, orientation, manifolds with boundary, integration of forms, Stokes' theorem.

\item \textbf{de Rham cohomology} (Lee 17) \\
de Rham groups, compactly supported de Rham groups, homotopy invariance, degree theory, Mayer-Vietoris sequences, Poincare duality.

\item \textbf{Hodge theory} (Warner 6) \\
Laplace-Beltrami operator, harmonic forms, elliptic regularity, Hodge decomposition.
\end{itemize}

\end{document}
















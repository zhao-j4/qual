\documentclass[reqno]{amsart}

\usepackage{amsfonts,latexsym,amsthm,amssymb,amsmath,amscd,euscript,bm}
\usepackage[sc]{mathpazo}
\usepackage[margin = 2cm]{geometry}
\usepackage{enumitem}
\usepackage{hyperref}
% sets numbering of enumerate to a, b, c, ...
\renewcommand{\theenumi}{\alph{enumi}}

% Theorems, propositions, etc.
\newtheorem{theorem}{Theorem}
\newtheorem{proposition}[theorem]{Proposition}
\newtheorem{lemma}[theorem]{Lemma}
\newtheorem{corollary}[theorem]{Corollary}

\theoremstyle{definition}
\newtheorem{definition}[theorem]{Definition}
\newtheorem*{claim}{Claim}

\theoremstyle{remark}
\newtheorem*{remark}{Remark}
\newtheorem*{notation}{Notation}

\usepackage{tikz-cd}

\usepackage{thmtools}
\usepackage[framemethod=TikZ]{mdframed}
	\mdfdefinestyle{mdrecbox}
		{%
			linewidth=0.5pt,
			skipabove=12pt,
			frametitleaboveskip=5pt,
			frametitlebelowskip=0pt,
			skipbelow=2pt,
			frametitlefont=\bfseries,
			innertopmargin=4pt,
			innerbottommargin=8pt,
			nobreak=true,
		}
	\declaretheoremstyle
		[
			headfont=\bfseries,
			mdframed={style=mdrecbox},
			headpunct={\\[3pt]},
			postheadspace={0pt},
		]
		{thmrecbox}
\newcounter{problem}[section]	\declaretheorem[style=thmrecbox,name=Problem, numberlike=problem]{statement}


% Solution environment
\newenvironment{solution}
	{
		\begin{proof}[Solution]}{\end{proof}
	}


% Math blackboard font
\newcommand{\nc}{\newcommand}
\nc{\on}[1]{\operatorname{#1}}

\nc{\R}{\mathbb R}
\nc{\C}{\mathbb C}
\nc{\Q}{\mathbb Q}
\nc{\Z}{\mathbb Z}
\nc{\N}{\mathbb N}
\nc{\HH}{\mathbb H}
\nc{\DD}{\mathbb D}
\nc{\TT}{\mathbb T}
\nc{\EE}{\mathbb E}
\nc{\PP}{\mathbb P}

\nc{\cH}{\mathcal H}
\nc{\cA}{\mathcal A}
\nc{\cM}{\mathcal M}
\nc{\cR}{\mathcal R}
\nc{\cB}{\mathcal B}
\nc{\cG}{\mathcal G}
\nc{\cD}{\mathcal D}
\nc{\cS}{\mathcal S}
\nc{\cF}{\mathcal F}
\nc{\cL}{\mathcal L}
\nc{\cE}{\mathcal E}

\nc{\diam}{\operatorname{diam}}
\nc{\supp}{\operatorname{supp}}
\nc{\vol}{\operatorname{vol}}
\nc{\im}{\operatorname{im}}
\nc{\loc}{\text{loc}}

% Why the f*** would you ever use \epsilon
\renewcommand{\epsilon}{\varepsilon}
\renewcommand{\emph}{\textsc}
\renewcommand{\Re}{\operatorname{Re}}
\renewcommand{\Im}{\operatorname{Im}}
%inverse Fourier transform widecheck

\let\vec\mathbf

% Title: change problem set number as needed
\title
{
	\emph{Hodge theory}
} 

\author{Jason Zhao}
\date{\today}

\begin{document}
\maketitle
\tableofcontents

\section{Preliminaries}

\subsection{Linear algebra}

Let $(V, \langle \cdot, \cdot \rangle)$ be a real inner product space of dimension $n$ with a choice of orientation $e_1 \wedge \cdots \wedge e_n$. The \emph{Hodge star operator} $\star : \bigwedge^k V \to \bigwedge^{n - k} V$
	\[ \star (e_{i_1} \wedge \cdots \wedge e_{i_k}) = (-1)^\sigma e_{j_1} \wedge \cdots \wedge e_{j_{n - k}}. \]
	
\begin{lemma}
	$\star \star = (-1)^{n (n - k)}$.
\end{lemma}

\subsection{Manifolds}

We endow the space of $k$-forms with the inner product
	\[ (\alpha, \beta) := \int_M \alpha \wedge \star \beta = \int_M \langle \alpha, \beta \rangle d \vol \]
It is easy to see that this is positive definite, 
	\[ (\omega, \omega) = \int_M \langle \omega, \omega \rangle d \vol \]
is non-negative and zero if and only if $\omega = 0$ by continuity. For symmetry, observe that $\langle \omega, \eta \rangle = \langle \star \omega, \star \eta \rangle$. Therefore
	\[ (\omega, \eta) = \int_M \langle \omega, \eta \rangle d \vol = \int_M \langle \star \omega, \star \eta \rangle d \vol \]

\section{Hodge decomposition theorem}

Denote $\cH^k (M)$ the space of harmonic $k$ forms. Then

\begin{theorem}[Hodge decomposition theorem]
	Let $M$ be a compact, oriented Riemannian manifold of dimension $n$. Then the space of differential forms admits the following orthogonal decompositions
		\begin{align*}
			\Omega^k (M) 
				&= \im \Delta \oplus \ker \Delta \\
				&= \im d \delta  \oplus \im \delta d \oplus \ker \Delta\\
				& = d (\Omega^{k - 1} (M)) \oplus \delta (\Omega^{k + 1} (M)) \oplus \ker \Delta.
		\end{align*}		
\end{theorem}

\begin{proof}
	Choose an orthonormal basis 
\end{proof}

\end{document}